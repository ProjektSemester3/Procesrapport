\chapter*{Projektgennemførelse}
\section*{Gruppedannelse}
Allerede inden semesterstart, uge 35, var gruppen dannet. Fælles værdier og enighed om samarbejde på baggrund af erfaringer fra 2. semesters projekt blev grundlaget for dannelsen af gruppen.

Det har i processen omkring gruppedannelse været diskuteret om gruppen bestod af nok medlemmer ift. det antal gruppemedlemmer ASE anbefalder til et 3. semesters projekt. På den ene side kan det være fordelagtigt at være en mindre gruppe fordi der vil være færre synspunkter at tage hensyn til, og dermed større mulighed for fælles enighed. En ulempe kan være at arbejdsbyrden per gruppemedlem bliver større. Disse overvejelser indgik i gruppens dannelse.\\

\section*{Samarbejdsaftale}
Gruppens samarbejdsaftale (reference til bilag) har været et godt sted at enes om forventinger og værktøjer, men har også været et dokument under forandring og et dokument som sjældent har været henvist til. 

Relevansen af samarbejdsaftalen består således først og fremmest i at være enige om visse retningslinjer og krav til hinanden, men også at have et dokument at henvise til i kraft af uenigheder gruppemedlemmer imellem.

\section*{Arbejdsfordeling}
Gruppen har gjort brug af de funktioner Scrum tilbyder. Hvert gruppemedlem har derfor haft mulighed for selv at vælge og tilrettelægge opgaver. Initiativ, tillid og selvorganisering har derfor været nøgleord for arbejdsfordelingen ved brug af Scrum.

Indtil implementeringsfasen var der en god fordeling ift. at arbejde med det kendte og det ukendte. F.eks udarbejdede HVB et IBD (reference til ordliste) selv om det er hardwarearkitektur.
%Tilføj fra implementeringsfasen og frem.

\section*{Planlægning}
%Hjælp

\section*{Projektledelse}
Misforståelser omkring Scrum Master rollen i starten gjorde, at gruppen blev hierakisk opdelt. Dette blev hurtigt vendt til en flad organisering hvor alle gruppemedlemmer var aktive beslutningstagere. På den måde blev ansvaret fordelt og inddragede hele gruppen som gjorde sammenholdet stærkere.

\section*{Projektadministration}
%Hjælp

\section*{Udviklingsforløb}
%Hele gruppen skal med ind over

\section*{Møder}

\section*{Konflikthåndtering}

\section*{Opnåede erfaringer}

\section*{Fremtidigt arbejde}