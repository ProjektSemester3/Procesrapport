\section(Forord}
Dette er procesrapporten for semesterprojekt 3 på retningerne IKT, EE og E ved Aarhus University, School of Engineering. Rapporten er skrevet af gruppe 10, som
har haft Søren Hansen som vejleder. Afleveringsdato for procesrapporten er 20. December 2016, og bedømmelse er den 18. Januar 2017.
Processrapporten indeholder gruppens overvejelser og refleksioner omkring processen for semesterprojektet, og er skrevet på baggrund af de erfaringer
gruppemedlemmerne har gjort sig under projektet.    

\section{Indledning}
Der er i dette semesterprojekt i modsætningen til tidligere blevet lagt meget vægt på proccesen for projektet. Gruppemedlemmerne har derfor været nød til at 
reflektere over hvilke processmæssige udfordringer de er blevet stillet overfor, hvilket har været en ny måde at gribe projektet an på.
Gruppen har især haft fokus på scrum udviklingsmetoden, da det er første gang der bliver stiftet bekendtskab med denne metode i praksis. Denne udviklingsmetode
gav gruppen en række nye værktøjer til styring af projekter, men har på nogle punkter været svær at implementere. I denne procesrapport vil der derfor være et 
særligt fokus på gruppens arbejde med scrum som udviklingsmetode, og hvordan gruppens til- og fravalg af elementer fra scrum påvirkede projektets forløb.
Der vil også blive fokuseret på hvilke konflikter og problemer gruppen stødte på undervejs i projektet. Herunder hvilke beslutninger der blev taget for at 
løse disse udfordringer, samt gruppens refleksioner omkring disse.         

\section{Projekt administration og ledelse}
Gruppen blev enige om fra starten at køre med en flad struktur, hvor alle påtog sig opgaven som productowner, og at der dermed ikke var nogen topstyring.  
Det var dog Scrum Masterens opgave at sørge for at agile boardet blev opdateret, og at gruppemedlemmerne fik skrevet logbøger. Han fik dermed en "lederagtig"
position i gruppen, som førte til misforståelser omkring Scrum Master rollen der gjorde, at gruppen blev hierarkisk opdelt. Dette blev hurtigt taget op på et 
retrospektiv, og vendt tilbage til en flad organisering, hvor alle gruppemedlemmer var aktive beslutningstagere. På den måde blev ansvaret fordelt og inddragede hele gruppen 
som gjorde sammenholdet stærkere. Denne flade struktur i gruppen fungerede godt i de første par sprints, og folk var gode til tage initiativ så opgaverne på
agile board blev løst indenfor deadline. Den fejlede dog i sprint 4 og 5, hvor mange af opgaverne på agile board ikke blev taget frivilligt. Nogle af gruppemedlemmerne
slap afsted med ikke at bidrage særlig meget til projektet og det blev tydeligt at gruppen havde brug for en leder, der kunne uddelegere arbejdet
og holde medlemmerne ansvarlige for at løse deres opgave til tiden. Fra sprint 6 blev der derfor udpeget en leder, der skulle sørge for at gruppen kunne 
færdiggøre projektet til tiden. Denne struktur gik dog på kompromis med scrum, hvor teamet ikke er hierarkisk opdelt. Gruppen følte dog at denne ændring var en
nødvendighed for det videre arbejde, da der tydeligvis manglede den fornødne selvdiciplin og ansvarsfølelse til at arbejde med en flad stuktur.       
 

\section{Gruppedynamik}
\subsection{Gruppefølelse}
Som nævnt blev gruppen dannet allerede inden semesterstart, og der var en opfattelse af gruppen som "stærk", med engagerede medlemmer der var fagligt dygtige.
Dette gav gruppen et godt udgangspunkt for et vellykket semesterprojekt, hvilket bidrog til en god stemning, og et godt sammenhold. Dette fortsatte i de første
3 sprints, hvor der var god mødediciplin og hvor gruppen var samlet omkring de opgaver der skulle løses. Gruppen mødes jævnligt til stå-op/arbejdsmøderne,
hvilket også bidrog til at at gruppen følte sig samlet omkring de udfordringer vi stod overfor, og der var generelt god kommunikation internt. Dog skete der et 
stort skift i gruppedynamikken efter efterårsferien. Og i de følgende sprints, 4 og 5, opstod der nærmest en opløsning af gruppefølelsen. Kommunikationen begyndte 
at blive dårlig, og visse gruppemedlemmer mødte ikke op til stå-op møderne, hvilket smittede af på de andre medlemmer. Til sidst var gruppen stort set splittet
op i mindre delenheder der hver tog sig af sit eget, og der var ingen følelse af sammenhold. F.eks. arbejdede software gruppen med SPI og GUI, og hardware 
tog sig af sensorer og motorer, uden at der var indbyrdes kommunikation eller overblik over projektet som helhed. Gruppen var dårlig til at håndtere disse 
problemer, og i flere uger blev denne situation uændret, hvilket havde en negativ effekt på gruppefølelsen. Dette blev heller ikke bedre af at et gruppemedlem 
forlod gruppen. Det var først i sprint 5, at disse problemer blev taget op med vejleder, og det blev besluttet at der skulle føres en stram linje mht. mødedeltagelse
på stå-op møderne. Der blev i det hele taget rusket godt op i gruppen af scrum master, og de resterende medlemmer begyndte at komme på rette spor igen. Set i bagspejlet
skulle der klart havde været taget hånd om den manglende gruppefølelse tidligere. Det virker klart at efterårsferien var med til at stoppe den ellers
gode udvikling gruppen var i. I fremtiden vil det være godt at sikre sig at en pause i skolegangen ikke bliver en hindring for projektet, og man kunne evt. 
have fortsat kommunikationen hen over ferien, og givet medlemmerne nogle lektier for. Desuden ville det måske være godt at holde et møde straks efter ferien
for at sikre sig, at man har et godt udgangspunkt for det fortsatte arbejde.  

\section{Værktøjers brug}
\subsection{Agile board}
Projektet blev administreret med scrum-værktøjet agile board, som gav et godt overblik over opgaverne for hvert sprint, og status på disse opgaver
i løbet af sprintet. Opgaverne blev lagt op på på agile boardet ved hvert sprintplanlægningsmøde, og hvert gruppemedlem var selv ansvarlig for at skrive sig på
opgaver. Dette var en god måde at uddelegere arbejdet på, og gruppemedlemmerne blev motiveret af at det var meget klart hvem der påtog sig opgaver i gruppen.
Dette fungerede rigtig godt i de første 3 sprints, hvor begejstringen for dette nye værktøj også var tydeligt i gruppen. Gruppen fik på disse sprints, de 
fleste opgaver løst, og boardet blev brugt flittigt. Men i de senere sprints blev agile board brugt mindre og mindre, og til sidste stod gruppen i den situation
at boardet stort set ikke blev opdateret. Der var mange opgaver som var løst, men som ikke var blevet opdateret. Dette gav et forkert billede af 
gruppens status, og bidog til forvirring og en følelse af uoverskuelighed. Det endte også med at sprint 4 og 5 blev forlænget flere gange, da der var mange 
uløste opgaver på agile board, og der var ikke overskud til at få ryddet op i opgaverne og starte et nyt sprint. Til sidst i sprint 5 blev det dog for uoverskueligt
og det blev besluttet at få ryddet agile board og starte på en frisk. Det gav gruppen det nødvendige overblik og fokus der skulle til for at komme på rette
spor igen. Gruppen lærte at Agile board på mange måder giver et godt billede og gruppens situation. Et struktureret agile board, som hele tiden bliver opdateret 
er meget vigtig for et godt procesforløb. Et dårligt overblik på agile board kan endda være med til at demotivere gruppen, fordi man mister overblikket.  
 
\subsection{Logbog}
Et andet værktøj der blev brugt var oprettelsen af en wiki hvori gruppemedlemmerne kunne føre logbog for deres arbejde. Her blev der givet en mere detaljeret 
beskrivelse af løste opgaver, samt eventuelle problematikker. Disse logbøger var tilgængelige for alle medlemmer i gruppen, hvilket hjalp med at give indsigt i 
hinandens arbejde. Disse logbøger blev også brugt flittigt i starten af projektet, men knap så meget senere. Enkelte gruppemedlemmer holdte helt op med at 
føre logbog, og der blev ikke tjekket op på det af scrum master. Gruppen havde mistet overskuddet til at tage sig af disse "småting", og det blev derfor
nedprioriteret, dog uden den helt store konsekvens for projektet.

\subsection{Github}
Github blev brugt til at gemme dokumenter, og holde styr på gruppens filer. De fleste gruppemedlemmer kendte dog ikke til git inden projektet, 
så dette gav nogle problemer især i starten af projektet med bl.a at merge filer. Overordnet var det et godt værktøj, der gjorde at gruppen for det meste havde
et overblik over hvilke dokumenter der var blevet færdiggjort. De enkelte medlemmer kunne nemt klone de forskellige reporsitories, og på den måde undgik gruppen
at det lå filer lokalt på medlemmers PCer, alle dokumenter var altid tilgængelige for alle.  

\subsection{Latex}
Enkelte medlemmer i gruppen havde et stort ønske om at anvende Latex til rapportskrivning. De havde i tidligere semesterprojekter brugt meget tid på at
samle dokumenter med word, og Latexs "include" funktion skulle gøre dette meget nemmere. Desuden er Latex et værktøj som anvendes meget på tekniske studier
og det ville derfor gavne hele gruppen at få kendskab til Latex. Det viste sig dog at Latex blev en af de store udfordringer, da det kræver en del
tid at sætte sig ind i hvordan det virker, noget som for de fleste gruppemedlemmer ikke blev prioriteret særlig højt. Dette betød at der kun var et medlem som havde 
styr på latex, mens de andre var i vildrede og derfor var afhængige af at dette ene medlem tog sig af de problemer der opstod. Dette var naturligvis ikke 
optimalt, og gav frustrationer for det medlem der stod for at rette alle de andres Latexfejl, hvilket kunne være meget tidskrævende. Gruppen valgte som løsning 
at et mere medlem skulle specialisere sig i Latex, og dermed også have overblik over projektrapporten og dokumentation. På denne måde sikredes det at der var 
styr på projekt dokumentation, et område som havde været plaget af manglende overblik. Gruppen lærte her at hvis der skal inddrages et nyt værktøj i et 
semesterprojekt, så skal der sættes tid af til at alle kan sættes sig grundigt ind i hvordan det virker. Man kan ikke satse på at det fungere hvis kun enkelte 
har et godt kendskab til værktøjet. 

\subsection{Slack}
Som kommunnikationsværktøj brugte gruppen slack. Dette værktøj kan integreres med andre værktøjet som github og google kalender, og blev brugt til 
sygemelding, mødeindkaldelser, generelle spørgsmål og beskeder fra scummaster. Det var allerede blevet brugt af nogle af gruppens medlemmer, og det var 
også disse som i starten brugte det mest. De øvrige medlemmer tog lidt tid om at komme igang, og det førte til noget frustration over manglende kommunikation.
Især scrum master var ret irriteret over manglende feedback på de beskeder og spørgsmål han lagde op, og det blev dog også hurtigt taget op på et retrospektiv.
Her blev det stillet som krav, at alle skulle downloade slack-appen til mobilen, og desuden skulle der markeres når man havde læst en besked. 
Scrum master fik desuden sin egen kanal på slack, så hans beskeder ikke forsvandt i mængden, og de andre medlemmer viste hvor de skulle finde de vigtigste 
beskeder. Slack fungerede i nogle perioder fint, men der var også perioder hvor folk ikke var online så ofte. Så et super effektivt beskedsystem blev det aldrig, 
men det var godt til have et sted hvor man kunne stille spørgsmål, og i øvrigt give nogle fælles beskeder.  
      
